\documentclass[11pt]{article}
\usepackage{blindtext, rotating, color}
\begin{document}
\begin{center}
\textbf{Workflow}

\emph{(Using the pillbox cavity model as an example.)}
\end{center}
\begin{turn}{45}
\begin{minipage}{0.2\linewidth}
\textcolor[rgb]{0.50,0.50,1.00}{Python FreeCAD}
\end{minipage}
\end{turn}
\begin{minipage}{0.8\linewidth}
Write a base geometry in Python and define geometric parameter sweeps (\verb|pillbox_cavity.py|)
\vspace{10mm}
\end{minipage}

\begin{turn}{45}
\begin{minipage}{0.2\linewidth}
\textcolor[rgb]{0.50,0.50,1.00}{Text files}

\textcolor[rgb]{0.50,0.50,1.00}{written in GdfidL input language.}
\end{minipage}
\end{turn}
\begin{minipage}{0.6\linewidth}
Define the volume of the mesh. (\verb|mesh_definition.txt|)
\vspace{10mm}
\end{minipage}

\begin{turn}{45}
\begin{minipage}{0.2\linewidth}
\textcolor[rgb]{0.50,0.50,1.00}{MATLAB}
\end{minipage}
\end{turn}
\begin{minipage}{0.8\linewidth}
Write a file to define the base simulation setup, the location and settings of any signal ports, the links between the STL files and the material names used in the EM simulation, and any material sweeps or simulation parameter sweeps. (\verb|Testing_pillbox_cavity.m|)
\vspace{10mm}
\end{minipage}

\begin{turn}{45}
\begin{minipage}{0.2\linewidth}
\textcolor[rgb]{0.50,0.50,1.00}{MATLAB}
\end{minipage}
\end{turn}
\begin{minipage}{0.8\linewidth}
\textbf{Run the framework.}

This will generate a report for each unique parameter set, and also summary reports for any sweeps.
\vspace{10mm}
\end{minipage}

There are other examples in the test suite/FreeCAD input folder.


\end{document} 